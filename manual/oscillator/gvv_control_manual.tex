\documentclass[journal,12pt,twocolumn]{IEEEtran}
%\usepackage{setspace}
\usepackage{gensymb}
%\doublespacing
%\singlespacing

%\usepackage{graphicx}
%\usepackage{amssymb}
%\usepackage{relsize}
\usepackage[cmex10]{amsmath}
%\usepackage{amsthm}
%\interdisplaylinepenalty=2500
%\savesymbol{iint}
%\usepackage{txfonts}
%\restoresymbol{TXF}{iint}
%\usepackage{wasysym}
\usepackage{amsthm}
%\usepackage{iithtlc}
\usepackage{mathrsfs}
\usepackage{txfonts}
\usepackage{stfloats}
\usepackage{bm}
\usepackage{cite}
\usepackage{cases}
\usepackage{subfig}
%\usepackage{xtab}
\usepackage{longtable}
\usepackage{multirow}
%\usepackage{algorithm}
%\usepackage{algpseudocode}
\usepackage{enumitem}
\usepackage{mathtools}
\usepackage{tikz}
\usetikzlibrary{shapes,arrows}
\usepackage{circuitikz}
\usepackage{verbatim}
%\usepackage{tfrupee}
%\usepackage[breaklinks=true]{hyperref}
%\usepackage{stmaryrd}
%\usepackage{tkz-euclide} % loads  TikZ and tkz-base
%\usetkzobj{all}
\usepackage{listings}
    \usepackage{color}                                            %%
    \usepackage{array}                                            %%
    \usepackage{longtable}                                        %%
    \usepackage{calc}                                             %%
    \usepackage{multirow}                                         %%
    \usepackage{hhline}                                           %%
    \usepackage{ifthen}                                           %%
  %optionally (for landscape tables embedded in another document): %%
    \usepackage{lscape}     
\usepackage{multicol}
\usepackage{chngcntr}
%\usepackage{enumerate}

%\usepackage{wasysym}
%\newcounter{MYtempeqncnt}
\DeclareMathOperator*{\Res}{Res}
%\renewcommand{\baselinestretch}{2}
\renewcommand\thesection{\arabic{section}}
\renewcommand\thesubsection{\thesection.\arabic{subsection}}
\renewcommand\thesubsubsection{\thesubsection.\arabic{subsubsection}}

\renewcommand\thesectiondis{\arabic{section}}
\renewcommand\thesubsectiondis{\thesectiondis.\arabic{subsection}}
\renewcommand\thesubsubsectiondis{\thesubsectiondis.\arabic{subsubsection}}

% correct bad hyphenation here
\hyphenation{op-tical net-works semi-conduc-tor}
\def\inputGnumericTable{}                                 %%

%\lstset{
%language=C,
%frame=single, 
%breaklines=true,
%columns=fullflexible
%}
%\lstset{
%language=tex,
%frame=single, 
%breaklines=true
%}

\begin{document}
%


\newtheorem{theorem}{Theorem}[section]
\newtheorem{problem}{Problem}
\newtheorem{proposition}{Proposition}[section]
\newtheorem{lemma}{Lemma}[section]
\newtheorem{corollary}[theorem]{Corollary}
\newtheorem{example}{Example}[section]
\newtheorem{definition}[problem]{Definition}
%\newtheorem{thm}{Theorem}[section] 
%\newtheorem{defn}[thm]{Definition}
%\newtheorem{algorithm}{Algorithm}[section]
%\newtheorem{cor}{Corollary}
\newcommand{\BEQA}{\begin{eqnarray}}
\newcommand{\EEQA}{\end{eqnarray}}
\newcommand{\define}{\stackrel{\triangle}{=}}
\bibliographystyle{IEEEtran}
%\bibliographystyle{ieeetr}
\providecommand{\mbf}{\mathbf}
\providecommand{\pr}[1]{\ensuremath{\Pr\left(#1\right)}}
\providecommand{\qfunc}[1]{\ensuremath{Q\left(#1\right)}}
\providecommand{\sbrak}[1]{\ensuremath{{}\left[#1\right]}}
\providecommand{\lsbrak}[1]{\ensuremath{{}\left[#1\right.}}
\providecommand{\rsbrak}[1]{\ensuremath{{}\left.#1\right]}}
\providecommand{\brak}[1]{\ensuremath{\left(#1\right)}}
\providecommand{\lbrak}[1]{\ensuremath{\left(#1\right.}}
\providecommand{\rbrak}[1]{\ensuremath{\left.#1\right)}}
\providecommand{\cbrak}[1]{\ensuremath{\left\{#1\right\}}}
\providecommand{\lcbrak}[1]{\ensuremath{\left\{#1\right.}}
\providecommand{\rcbrak}[1]{\ensuremath{\left.#1\right\}}}
\theoremstyle{remark}
\newtheorem{rem}{Remark}
\newcommand{\sgn}{\mathop{\mathrm{sgn}}}
\providecommand{\abs}[1]{\left\vert#1\right\vert}
\providecommand{\res}[1]{\Res\displaylimits_{#1}} 
\providecommand{\norm}[1]{\left\lVert#1\right\rVert}
%\providecommand{\norm}[1]{\lVert#1\rVert}
\providecommand{\mtx}[1]{\mathbf{#1}}
\providecommand{\mean}[1]{E\left[ #1 \right]}
\providecommand{\fourier}{\overset{\mathcal{F}}{ \rightleftharpoons}}
%\providecommand{\hilbert}{\overset{\mathcal{H}}{ \rightleftharpoons}}
\providecommand{\system}{\overset{\mathcal{H}}{ \longleftrightarrow}}
	%\newcommand{\solution}[2]{\textbf{Solution:}{#1}}
\newcommand{\solution}{\noindent \textbf{Solution: }}
\newcommand{\cosec}{\,\text{cosec}\,}
\providecommand{\dec}[2]{\ensuremath{\overset{#1}{\underset{#2}{\gtrless}}}}
\newcommand{\myvec}[1]{\ensuremath{\begin{pmatrix}#1\end{pmatrix}}}
\newcommand{\mydet}[1]{\ensuremath{\begin{vmatrix}#1\end{vmatrix}}}
%\numberwithin{equation}{section}
\numberwithin{equation}{subsection}
%\numberwithin{problem}{section}
%\numberwithin{definition}{section}
\makeatletter
\@addtoreset{figure}{problem}
\makeatother
\let\StandardTheFigure\thefigure
\let\vec\mathbf
%\renewcommand{\thefigure}{\theproblem.\arabic{figure}}
\renewcommand{\thefigure}{\theproblem}
%\setlist[enumerate,1]{before=\renewcommand\theequation{\theenumi.\arabic{equation}}
%\counterwithin{equation}{enumi}
%\renewcommand{\theequation}{\arabic{subsection}.\arabic{equation}}
\def\putbox#1#2#3{\makebox[0in][l]{\makebox[#1][l]{}\raisebox{\baselineskip}[0in][0in]{\raisebox{#2}[0in][0in]{#3}}}}
     \def\rightbox#1{\makebox[0in][r]{#1}}
     \def\centbox#1{\makebox[0in]{#1}}
     \def\topbox#1{\raisebox{-\baselineskip}[0in][0in]{#1}}
     \def\midbox#1{\raisebox{-0.5\baselineskip}[0in][0in]{#1}}
\vspace{3cm}
\title{
	\logo{
Control Systems
	}
}
\author{ G V V Sharma$^{*}$% <-this % stops a space
	\thanks{*The author is with the Department
		of Electrical Engineering, Indian Institute of Technology, Hyderabad
		502285 India e-mail:  gadepall@iith.ac.in. All content in this manual is released under GNU GPL.  Free and open source.}
	
}	
%\title{
%	\logo{Matrix Analysis through Octave}{\begin{center}\includegraphics[scale=.24]{tlc}\end{center}}{}{HAMDSP}
%}
% paper title
% can use linebreaks \\ within to get better formatting as desired
%\title{Matrix Analysis through Octave}
%
%
% author names and IEEE memberships
% note positions of commas and nonbreaking spaces ( ~ ) LaTeX will not break
% a structure at a ~ so this keeps an author's name from being broken across
% two lines.
% use \thanks{} to gain access to the first footnote area
% a separate \thanks must be used for each paragraph as LaTeX2e's \thanks
% was not built to handle multiple paragraphs
%
%\author{<-this % stops a space
%\thanks{}}
%}
% note the % following the last \IEEEmembership and also \thanks - 
% these prevent an unwanted space from occurring between the last author name
% and the end of the author line. i.e., if you had this:
% 
% \author{....lastname \thanks{...} \thanks{...} }
%                     ^------------^------------^----Do not want these spaces!
%
% a space would be appended to the last name and could cause every name on that
% line to be shifted left slightly. This is one of those "LaTeX things". For
% instance, "\textbf{A} \textbf{B}" will typeset as "A B" not "AB". To get
% "AB" then you have to do: "\textbf{A}\textbf{B}"
% \thanks is no different in this regard, so shield the last } of each \thanks
% that ends a line with a % and do not let a space in before the next \thanks.
% Spaces after \IEEEmembership other than the last one are OK (and needed) as
% you are supposed to have spaces between the names. For what it is worth,
% this is a minor point as most people would not even notice if the said evil
% space somehow managed to creep in.
% The paper headers
%\markboth{Journal of \LaTeX\ Class Files,~Vol.~6, No.~1, January~2007}%
%{Shell \MakeLowercase{\textit{et al.}}: Bare Demo of IEEEtran.cls for Journals}
% The only time the second header will appear is for the odd numbered pages
% after the title page when using the twoside option.
% 
% *** Note that you probably will NOT want to include the author's ***
% *** name in the headers of peer review papers.                   ***
% You can use \ifCLASSOPTIONpeerreview for conditional compilation here if
% you desire.
% If you want to put a publisher's ID mark on the page you can do it like
% this:
%\IEEEpubid{0000--0000/00\$00.00~\copyright~2007 IEEE}
% Remember, if you use this you must call \IEEEpubidadjcol in the second
% column for its text to clear the IEEEpubid mark.
% make the title area
%\maketitle
\newpage
\tableofcontents
\bigskip
\renewcommand{\thefigure}{\theenumi}
\renewcommand{\thetable}{\theenumi}
%\renewcommand{\theequation}{\theenumi}
%\begin{abstract}
%%\boldmath
%In this letter, an algorithm for evaluating the exact analytical bit error rate  (BER)  for the piecewise linear (PL) combiner for  multiple relays is presented. Previous results were available only for upto three relays. The algorithm is unique in the sense that  the actual mathematical expressions, that are prohibitively large, need not be explicitly obtained. The diversity gain due to multiple relays is shown through plots of the analytical BER, well supported by simulations. 
%
%\end{abstract}
% IEEEtran.cls defaults to using nonbold math in the Abstract.
% This preserves the distinction between vectors and scalars. However,
% if the journal you are submitting to favors bold math in the abstract,
% then you can use LaTeX's standard command \boldmath at the very start
% of the abstract to achieve this. Many IEEE journals frown on math
% in the abstract anyway.
% Note that keywords are not normally used for peerreview papers.
%\begin{IEEEkeywords}
%Cooperative diversity, decode and forward, piecewise linear
%\end{IEEEkeywords}
% For peer review papers, you can put extra information on the cover
% page as needed:
% \ifCLASSOPTIONpeerreview
% \begin{center} \bfseries EDICS Category: 3-BBND \end{center}
% \fi
%
% For peerreview papers, this IEEEtran command inserts a page break and
% creates the second title. It will be ignored for other modes.
%\IEEEpeerreviewmaketitle
\begin{abstract}
This manual is an introduction to control systems based on GATE problems.Links to sample Python codes are available in the text.  
\end{abstract}

%\begin{lstlisting}
%svn co https://github.com/gadepall/school/trunk/control/codes
%\end{lstlisting}
%\section{Bode Plot}
%\begin{enumerate}[label=\thesection.\arabic*.,ref=\thesection.\theenumi]
%\numberwithin{equation}{enumi}

%\end{enumerate}
\section{Stability}
\section{Routh Hurwitz Criterion}


\section{Compensators}
\section{Nyquist Plot}
\section{State space model}
\section{oscillator}
\begin{enumerate}[label=\thesection.\arabic*.,ref=\thesection.\theenumi]
\numberwithin{equation}{enumi}

Oscillators generate AC output (the waveform),without any external input \\
Resonant frequency, is the frequency at which oscillator oscillates, it depends on R/L/C components of the circuit it's been fed back through.\\
Oscillators work because they overcome the losses of their feedback circuit either in the form of a capacitor, inductor or both. In other words, an oscillator is a an amplifier which uses feedback that generates an output frequency without the use of an input signal.\\ %brief description
\newline

\item Draw the equivalent block diagram of an oscillator.\\
\solution
\begin{figure}[!ht]
    \begin{center}
		
		\resizebox{\columnwidth}{!}{\tikzset{
        amp/.style = {regular polygon, regular polygon sides=3,
              draw, fill=white, text width=1em,
              inner sep=1mm, outer sep=0mm,
              shape border rotate=-90},
        block/.style = {draw, rectangle,
            minimum height=1cm,
            minimum width=2cm},
        input/.style = {coordinate,node distance=1cm},
        output/.style = {coordinate,node distance=4cm},
        arrow/.style={draw, -latex,node distance=2cm},
        pinstyle/.style = {pin edge={latex-, black,node distance=2cm}},
        sum/.style = {draw, circle, node distance=1cm},
}

\begin{tikzpicture}[node distance=2.5cm,auto,>=latex']
  \node [input, name=input] {};
  \node [sum, right of=input] (sum) {};
  \node [amp, right of = sum] (block1) {$G$};
  \node [output, right of= block1] (output) {};
  \node [block, below of = block1] (block2) {$H$};
  \draw [->] (input) -- node {$V_i_n$} (sum);
  \draw [->] (sum) -- node {$V_i_n + HV_o_u_t$} (block1);
  \draw [->] (block1) -- node [name =y] {$V_o_u_t$} (output);
  \draw [->] (y) |- node [above,pos=0.79] {$V_o_u_t$} (block2) ;
  \draw [->] (block2) -| node  {$HV_o_u_t$} (sum) ;
\end{tikzpicture}
} %block diagram
	\end{center}
\caption{block diagram for oscillator}
\label{fig:ee18btech11019_hart_block}
\end{figure}
Fig. \label{fig:ee18btech11019_hart_block} shows the block diagram of the an Oscillator in Fig. \ref{fig:ee18btech11019_hart_block}.\\




\item Show that the gain of the oscillator is \\
\begin{align}
    G = \frac{V_{out}}{V_{in}} = \frac{A}{1 - AB}
\label{eq:ee18btech11019_gain}
\end{align}
%
\\
\solution From \ref{fig:ee18btech11019_hart_block}
Oscillators gain can be given as follows:\\
\begin{align}
    A(V_{in} + BV_{out}) =V_{out}\\
    A(V_{in} = (1-AB)V_{out}\\
    \frac{V_{out}}{V_{in}} = \frac{A}{1 - AB}
\end{align}
%
resulting in \eqref{eq:ee18btech11019_gain}.



\item State the condition for sustained oscillations. Justify.

\solution Condition for sustained oscillation is given by\\
\begin{align}
    AB = 1
\end{align}
Along with, total phase gain o the circuit should be 0 or 2$\pi$\\
\textbf{Justification:} as, when $ AB =1 $, gain becomes infinity, and theoretically we can get output, without actually providing input\\
Total phase gain should be so, as we want our signal to be in phase after every loop traversal.\\


\item Find $A$ and $B$.

\solution Consider the below circuit
\begin{figure}[!ht]
    \begin{center}
		\resizebox{\columnwidth}{!}{\tikzset{
 block/.style = {draw, rectangle,
            minimum height=1cm,
            minimum width=2cm},
ifnode/.style={rectangle,dashed,draw=black, top color=white, inner sep=1em,minimum width=4.9cm, minimum height=5cm, text centered, fill=yellow}  ,    
iffnode/.style={rectangle,dashed,draw=black, top color=white, inner sep=1em,minimum width=3.8cm, minimum height=3cm, text centered, fill=yellow} 
            }
\begin{circuitikz} 
\node[ifnode, label=below: Feedback $H(s)$ as load] (ifin) at (2,-2) {};
\node[iffnode, label= Equivalent for amplifier $G(s)$] (ifin) at (-2.5,-1) {};
\draw
(0,0) -- (4,0) node[label = $V_{out}$]
  to [european resistor,label=$Z_2$]  (4,-4)  -- (0,-4)node[ground] {} to[european resistor = $Z_1$] (0,-2) 
  to[european resistor = $Z_3$] (0,0) to[american resistor,label=$R_o$] (-4,0) to[american voltage source ,invert,label= $A_vV_{in}$](-4,-1) node[ground] {}
  (0,-2)--(-6,-2)--(-6,0) node[label = $V_f$];
  

\end{circuitikz}
}
		
	\end{center}
\caption{block diagram for oscillator}
\label{fig:ee18btech11019_block2}
\end{figure}
We know that feedback gain is B, i.e, $\frac{V_0}{V_f} = B$\\
Applying voltage divider rule we get
\begin{align}
    B = \frac{Z_1}{Z_1 + Z_3}
\end{align}
\begin{align}
    A = \frac{V_o}{V_{in}} = \frac{AZ_L}{R_o + Z_L}\\
\end{align}    
    where, $Z_L$ is equivalent load across output
\begin{align}    
    Z_L = \frac{(Z_1 + Z_3)Z_2}{Z_1+Z_2+Z_3}\\
\end{align}


\item Find the frequency of oscillation using the condition that $AB = 1$.

\solution For any LC oscillator, 
Now,we know that $AB = 1$ for sustained oscillations, putting the the above terms in the equation\\
on solving,\\
\begin{align}    
    AB = \frac{Z_1Z_2A}{(Z_1+Z_2+Z_3)R_o+ Z_2(Z_1+Z_3)}\\
\end{align}    
\textbf{Hartley oscillator}:\\
The Hartley oscillator is one of the classical LC feedback circuits,i.e feedback is made of LC components.Below here 
\begin{align}
    Z_1 = SL_1 (inductor)\\
    Z_2 = SL_2 (inductor)\\
    Z_3 = \frac{1}{SC} (capacitor)
\end{align}

putting that in and equating $AB=1$ we get,

\begin{align}
1 = \frac{S^{2}L_1L_2A}{(SL_1+SL_2+\frac{1}{SC})R_o+ SL_2(SL_1+\frac{1}{SC})}\\
S^{2}L_1L_2A = (SL_1+SL_2+\frac{1}{SC})R_o+ SL_2(SL_1+\frac{1}{SC})
\end{align}

As we need, to find frequency, put S =jw
\begin{align}
    \omega^{2}L_1L_2A = j(\omega L_1 + \omega L_2 -\frac{1}{\omega C})R_o -\omega L_2(\omega L_1 + \frac{1}{\omega C})
\end{align}
To satisfy the above equation, equating imaginary term to Zero.
\begin{align}    
    \omega L_1 + \omega L_2  = \frac{1}{\omega C}\\
    \omega = \frac{1}{\sqrt{(L_1+L_2)(C)}}\\
    f = \frac{1}{2\pi \sqrt{(L_1+L_2)(C)}}
\end{align}
\begin{align}
    B = \frac{Z_1}{Z_1 + Z_3} = \frac{Z_1}{Z_2}\\
      = \frac{L_1}{L_2}\\
    A =  \frac{L_2}{L_1} 
\end{align}
 
\item For Hartley oscillator frequency generated can be given as 
\begin{align}
    f = \frac{1}{2\pi\sqrt{(L_1 + L_2)C}}
    \label{eq:frequency}
\end{align}
Fig. \ref{fig:ee18btech11019_hart} shows a
Hartley oscillator built using opamp.\\

\begin{figure}[ht]
    \begin{center}
	    \resizebox{\columnwidth}{!}{\tikzset{
ifnode/.style={rectangle,dashed,draw=black, top color=white, inner sep=1em,minimum width=16cm, minimum height=5cm, text centered, fill=yellow}
}
\begin{circuitikz} [scale=2]
\node[ifnode, label=above:  Amplifier $G(s)$] (ifin) at (-2.7,0) {};
\node[ifnode, label=below:  Feedback $H(s)$] (ifin2) at (-2.8,-2.8) {};
\draw 
(0,0) node[op amp] (opamp) [scale=2] {}
(opamp.-) -- (-3,0.5) to [R=$R_1$] (-5,0.5) -- (-6,0.5) -- (-6,-2) to [L=$L_2$] (-4,-2)
-- (-1,-2) to[L=$L_1$] (1,-2) -- (1,0) {};
\node[draw,box] (A) at (1.5,-0.7) {$V_{out}$};
\node[draw,box] (B) at (-6.5,-3) {$V_{f}$};
\draw (-6,-2) -- (-6,-3)  to [C=$C$] (1,-3) -- (1,-2) 

(opamp.-) to[short,*-] ++(0,1) coordinate (leftC)
to[R=$R_2$] (leftC -| opamp.out)
to[short,-*] (opamp.out) to [short,-o] (1.5,0) to (1.5,-0.5) {}
(opamp.+) -- (-3,-0.5) -- (-5,-0.5) node[ground] [scale=2] {};
\draw (-2,-0.5) -- (-2,-2);
\draw (-6,-3) -- (-6.3,-3);


\end{circuitikz}
}
	\end{center}
\caption{Hartley oscillator}
\label{fig:ee18btech11019_hart}
\end{figure}
\newline
\item \textbf{Simulation:}\\
Taking the following values,and applying in \ref{eq:frequency} \\



\begin{tabular}{|c|c|}
\hline
Component & Value  \\
\hline
$R_1$         & 10K$\Omega$   \\
\hline
$R_2$         & 100K$\Omega$   \\
\hline
$R_3$         & $\sim$  \\
\hline
$L_1$         & $1 \mu H$     \\
\hline
$L_2$         & $1 \mu H$   \\
\hline
C         & 120 pF \\
\hline
\end{tabular}


We get f = 103 MHz\\
Feedback factor for Hartley given by:
\begin{align}
B =\frac{L_1}{L_2}= 1
\end{align}
W.K.T, $AB = 1$\\
$\therefore$ Minimum amplification Gain,A = 1\\
\end{enumerate}

%\section{Triangle}
%\subsection{Triangle Examples}
%\input{./triangle/tri_exam.tex} 
%\subsection{Triangle Exercises}
%\input{./triangle/tri_geo.tex} 
%%
%\section{Quadrilateral}
%\subsection{Quadrilateral Examples}
%\input{./quad/quad_geo_exam.tex} 
%\subsection{Quadrilateral Geometry}
%\input{./quad/quad_geo.tex} 
%
%\section{Constrained Optimization}
%%\subsection{Equality Constraint}
%\input{./chapters/line_exam.tex}
%\section{Convex Function}
%\input{./chapters/conv.tex}
%\section{Gradient Descent}
%\input{./chapters/grad_des.tex}
%\section{Lagrange Multipliers}
%\input{./chapters/lagrange.tex}
%\section{Quadratic Programming}
%\input{./chapters/qp.tex}
%\section{Semi Definite Programming}
%\input{./chapters/sdp.tex}
%\section{Linear Programming}
%\input{./chapters/lp_exam.tex}
%\section{Exercises}
%\input{./chapters/lp_exer.tex}
\end{document}

