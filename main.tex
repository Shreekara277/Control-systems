\documentclass{beamer}
\usepackage[utf8]{inputenc}
\usetheme{Madrid}

\title
{Control Systems}
\subtitle{IN GATE QUESTION , 2018- QUESTION 37}
\author{EE18BTECH11019, HARSH KABRA}
%\author{F}


\begin{document}
\frame{\titlepage}
\begin{frame}
\frametitle{PROBLEM STATEMENT}
\emph{Q)} Consider the linear system \emph{$\dot{x}$} =
\begin{bmatrix}
-1 & 0\\
0 & -2
\end{bmatrix}x , with initial condition\\x(0) =
\begin{bmatrix}
1\\
1
\end{bmatrix}. The solution x(t) for this system is: \\
\vspace{7mm}



(A) x(t) =
\begin{bmatrix}
e^{-t} & te^{-2t}\\
0 & e^{-2t}
\end{bmatrix}
\begin{bmatrix}
1\\
1
\end{bmatrix} \hfill 
(B) x(t) =
\begin{bmatrix}
e^{-t} & 0\\
0 & e^{2t}
\end{bmatrix}
\begin{bmatrix}
1\\
1
\end{bmatrix}
\\
\vspace{3mm}
(C) x(t) =
\begin{bmatrix}
e^{-t} & t^{2}e^{-2t}\\
0 & e^{-2t}
\end{bmatrix}
\begin{bmatrix}
1\\
1
\end{bmatrix} \hfill 
(D) x(t) =
\begin{bmatrix}
e^{-t} & 0\\
0 & e^{-2t}
\end{bmatrix}
\begin{bmatrix}
1\\
1
\end{bmatrix}
\\


\end{frame}
\begin{frame} 

\frametitle{Solution}
It is of the form \emph{$\dot{x}$ = Ax}. Therefore its solution is \\
\vspace{3mm}
\hspace{30 mm}x(t) = e^{At}x(0)\\
$e^{At}$ is my state transition matrix 
and is equal to\\
\hspace{30mm}
$\mathcal{L}$^{-1}$[sI - A]^{-1}$ \hspace{7mm} (derivation in last slide)\\
\hspace{5mm}
A = 
\begin{bmatrix}
-1 & 0\\
0 & -2
\end{bmatrix} \hspace{20mm}
$\therefore$ [sI - A] = 
\begin{bmatrix}
s+1 & 0\\
0 & s+2
\end{bmatrix}\\
\vspace{5mm}
\hspace{5mm}
\rightarrow Adj(sI -A) =
\begin{bmatrix}
s+2 & 0\\
0 & s+1
\end{bmatrix}\\
\vspace{5mm}
\hspace{5mm}
\rightarrow det(sI -A) = (s+1)(s+2)
\end{frame}
\begin{frame}
%\begin{center}
\frametitle{Solution}
\hspace{5mm}
$\therefore$ [sI - A]^{-1} = \frac{Adj(A)}{det(A)} = 
\begin{bmatrix}
\frac{1}{s+1} & 0\\
0 & \frac{1}{s+2}
\end{bmatrix} \\
\vspace{5mm} \hspace{5mm}
e^{At} = 
\begin{bmatrix}
\mathcal{L}^{-1}[\frac{1}{s+1}] & 0\\
0 & \mathcal{L}^{-1}[\frac{1}{s+2}]
\end{bmatrix} \\
\vspace{5mm} \hspace{5mm}
e^{At} = 
\begin{bmatrix}
e^{-t} & 0\\
0 & e^{-2t}
\end{bmatrix} \\
\vspace{5mm} \hspace{5mm}
\therefore x(t) = e^{At} x(0)   =    
\begin{bmatrix}
e^{-t} & 0\\
0 & e^{-2t}
\end{bmatrix}
\begin{bmatrix}
1\\
1
\end{bmatrix}
\\
\vspace{5mm} \hspace{20mm}
\rightarrow \emph{OPTION (D)}
%\end{center}
\end{frame}
\begin{frame}
\frametitle{Derivation of state transition matrix}
\hspace{25mm}
\rightarrow \dot{x} = Ax \\
\vspace{5mm} \hspace{25mm}
Taking Laplacian\\
\vspace{5mm} \hspace{20mm}
\rightarrow S.X(s) -x(0) = AX(s)\\
\vspace{5mm} \hspace{20mm}
\therefore X(s) = [SI -A]^{-1}x(0)\\
\vspace{5mm} \hspace{20mm}
X(t) = $\mathcal{L}$^{-1} [SI -A]^{-1}x(0)\\
\vspace{5mm} \hspace{20mm}
\rightarrow Here,  \mathcal{L}^{-1}$[SI -A]^{-1}$  is called the state transition matrix

\end{frame}

\end{document}

